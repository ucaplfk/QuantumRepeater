\section{\label{sec:level1}Introduction}
The concept of building a quantum network is based on having photonic channels connected via quantum repeaters. A quantum network has applications to provide long distance quantum communication, quantum key distribution and distributed quantum computing $^{[}$\citep{Sasaki2017QuantumHeading,Zwerger2018Long-RangeTransmission}$^{]}$. Quantum repeaters are required due to the attenuation of light over long distances. In classical communication signal attenuation can be rectified using an amplifier. However, unknown quantum state transfer is complicated by the no-cloning theorem $^{[}$\citep{Sangouard2011QuantumOptics,Wootters1982ACloned}$^{]}$.The key requirements of a quantum repeater include a sufficiently long memory, loss and operational error suppression, sufficiently fast communication rate and polynomial resource overhead $^{[}$\citep{Briegel1998QuantumCommunication,Duan2001Long-distanceOptics,Muralidharan2016OptimalCommunication}$^{]}$. The properties of Rydberg states, such as the long lifetime and strong dipole-dipole interactions, make Rydberg atoms a good candidate to provide a quantum memory and are an ideal resource of a controllable optical interface $^{[}$\citep{Saffman2010QuantumAtoms}$^{]}$. These properties also mean Rydberg atoms have applications in hybrid quantum information processing $^{[}$\citep{Petrosyan2009ReversibleEnsembles,Pritchard2014HybridResonator}$^{]}$.

