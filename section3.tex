\section{\label{sec:level1}Conclusion and Outlook}
Collective Rydberg state encoding of an atomic ensemble and the concept of building a qubit register, using internal atomic states, provides a promising proposal to generate entangled states $^{[}$\citep{Pedersen2009FewEncoding}$^{]}$. This protocol has the advantage of increased light coupling, highly directional photon emission and inclusion of ancilla qubits to complete quantum error correction or entanglement purification. Qubits in the qubit register have functions depending on the excited state lifetime such as storage or fast $\exp(-2\gamma t)$-rate emission.      

Experimental realisation of collective encoding, enabled by utilising the Rydberg blockade effect, enables single Rb ensemble $\mathcal{N}=1$ Fock state generation with a fidelity of 62 \% \citep{Ebert2014AtomicBlockade}. Furthermore, $\mathcal{N}=2$ preparation has a lower measured fidelity of 48 \%  but provides realisation of the enhanced Rabi frequency oscillation of $\sqrt{2}\Omega_{1}$. The coherence time of 2.63 ms for $\bar{N}=7.6$ collective encoding is measured in Ref. [\citen{Ebert2015CoherenceQubits}] using Ramsey interferometry. The coherence time is a function of $1/\bar{N}$ with the most dominant source of decoherence arising from atomic collisions. One suggested means to suppress atomic collisions is the use of an atomic lattice trap. 

Additionally, in Ref. [\citen{Ebert2015CoherenceQubits}] the Rydberg blockade effect enables collective encoding of two atomic ensembles with $\mu$m-range separation. From post-selection of preparing $\ket{\textbf{1}}_{c}$ and removing blow-away excitation realises perfect inter-ensemble Rydberg blockade. However, the 0.89 fidelity of preparing $\ket{\textbf{1}}_{c}\ket{\textbf{0}}_{c}$ is limited by the 0.52 fidelity of single ensemble $\ket{\textbf{1}}_{c}$ preparation. This is expected to be due to stronger interactions between atoms in a single ensemble compared to the inter-ensemble interaction with the same separation. Long range dipole-dipole interaction of pairs along $\vec{z}$ may reduce the effectiveness of the Rydberg blockade effect in the single ensemble. Another possible reason for the reduced fidelity is creation of energy level mixing of molecular resonances at short-ranges leading to the creation of resonant states which are not Rydberg blocked $^{[}$\citep{Derevianko2015EffectsBlockade}$^{]}$. 

Utilising the Rydberg blockade effect allows generation of various entangled states as presented in Ref. [\citep{Nielsen2010DeterministicProcessing}]. The application of the GHZ and cluster state generation for quantum networking is considered in Ref.[\citen{Wallnofer2016Two-dimensionalRepeaters}] for two-dimensional quantum repeater architectures. The benefit would be multi-party communication. Cluster state or GHZ state lattices are connected through Bell-state measurements. This protocol also enables error detection and the ability to apply corrections through Pauli operations if the error syndrome is unique. Furthermore, the approach interfacing atoms via a CPW cavity mode aims to increase the distance between atomic ensembles to mm-ranges $^{[}$\citep{Petrosyan2008QuantumResonators}$^{]}$. The use of a microwave-CPW enables entanglement generation through cavity mediated van der Waals interaction between Rydberg excited states. Ref. [\citen{Sarkany2015Long-rangeCavity}] highlights advantage of the cavity approach where insensitivity to thermal cavity photons allows operation at > 100 mK temperatures typically used experimentally to contain atoms in a dipole trap. 

In conclusion, the long lifetime and strong dipole interaction between Rydberg states make them ideal candidates to build a qubit memory for the proposed quantum repeater architectures such as the DLCZ protocol. The experimental demonstration of the intra-ensemble Rydberg blockade prepares collective states with a fidelity of 0.89. This is a significant step towards generating deterministic entanglement. Future near-term quantum networking experiments should include: exploring methods to increase the fidelity of single ensemble state preparation, investigate the fidelity of generating proposed Bell, GHZ and two-dimensional cluster state entanglement and implement the long-distance approach using a CPW. 






