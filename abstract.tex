\begin{abstract}
The aim of this report is investigate the use of Rydberg atoms in proposed quantum repeater protocols. Rydberg atoms are presented as a good candidate to build a quantum repeater memory register. Additionally, the interaction between Rydberg atoms make them a good source of entanglement generation for a quantum repeater network. Significant experimental developments are presented including collective encoding of atoms, the Rydberg blockade effect and controlled interfacing of Rydberg atoms with photons. Theoretical extensions of the experimental demonstration enable the creation of complex output entanglement states. Additionally, the theoretical proposal to interface Rydberg atoms via a cavity mode is reviewed which explores extending the entanglement distance between Rydberg atoms.  
\end{abstract}